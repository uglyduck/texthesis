\documentclass[12pt,twoside,a4paper]{report}
\usepackage{color}


\title{\textbf{Securing GNSS Receivers with a Density-based Clustering Algorithm}}
\date{June 20151}
\author{Rashedul Amin Thin}
\pagenumbering{gobble}

\begin{document}
 
\newpage
 \maketitle
This page is intentionally left blank.

\newpage
\begin{abstract}
Global Navigation Satellite Systems (GNSS) is in widespread use around the world for numerous purposes. Even though it was first developed for military purposes, nowadays, the civilian use has surpassed it by far. It has evolved to its finest state in recent days and still being developed further towards pinpoint accuracy. With all the improvements, several vulnerabilities have been discovered by researchers and exploited by the attackers. Several countermeasures have been and still being implemented to secure the GNSS. Studies show that GNSS-based receivers are still vulnerable to a very fundamentally simple, yet effective, attack; known as the replay attack. The replay attack is particularly harmful since the attacker could make the receiver calculate an inaccurate position, without even breaking the encryption or without employing any sophisticated technique. The Multiple Combinations of Satellites and Systems (MCSS) test is a powerful test against replay attacks on GNSS. However, detecting and identifying multiple attacking signals and determining the correct position of the receiver simultaneously remain as a challenge. In this study, after the implementation of MCSS test, a mechanism to detect the attacker controlled signals has been demonstrated. Furthermore, applying a clustering algorithm on the product of MCSS test, a method of correctly determining the position, nullifying the adversarial effects has also been presented in this report.
\end{abstract}
\pagenumbering{roman}
\newpage
\begin{chapter}
\title{Acknowledgement}
\maketitle
First of all, I would like to express heartfelt gratitude to Panagiotis Papadimitratos, Associate Professor at KTH, for providing the opportunity to conduct the master thesis under his supervision. During the whole period, he has always been a great mentor and source of boundless inspiration and support. Also, I am grateful to Kewei Zhang for his
generous help, encouragement and support.

I would also like to thank the Swedish Institute (SI) for funding my masters’ studies in Network Services and Systems at KTH. 
I wholeheartedly acknowledge the contribution of my university, KTH, for providing the resources to conduct the study. I recognize the contribution Kai Borre, Professor at Aalborg Universitet, Denmark, for the EASY suite and UNAVCO, a non-profit university-governed consortium, for providing the required data.

I would like to thank everyone who helped and supported during the whole period and especially to Nikolas Gkikas, Syeda Farhana Afroz, Mohammad Ahsan Adib Murad and Zakaria Habib. I would also like to thank my beloved family and friends for their support.
\end{chapter}


\end{document}